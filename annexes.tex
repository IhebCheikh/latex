\chapter*{Annexes}
\addcontentsline{toc}{chapter}{Annexes}
\markboth{Annexes}{}
\stepcounter{chapter}
\addtocontents{lot}{\vspace{3.8mm}}
\addtocontents{lof}{\vspace{3.8mm}}

%Mettez vos annexes ici...

%===================== ANNEXE 1 =====================%

\section{Annexe 1 : Classification de CV}

\begin{tcolorbox}[colback=gray!5,colframe=gray!60!black,title={Prompt  : CV Classification}, breakable]
\textbf{Objective:} Classify a CV and extract structured information from the extracted text.

\textbf{Context:}  
You are an assistant specialized in IT recruitment.  
Your task is to classify a CV and extract relevant information.

\textbf{Classification categories:}
\begin{itemize}
    \item Software Engineering
    \item Software Architect
    \item QA
    \item Support Engineering
    \item DevOps Engineering
    \item Project Manager
    \item Others (if none of the above categories apply)
\end{itemize}

\textbf{Output format (JSON):}  
Make sure that \texttt{years\_of\_experience} is a numeric value.

\begin{lstlisting}[language=json,frame=single,basicstyle=\ttfamily\footnotesize]
{
  "name": "Candidate's name",
  "classification": "Corresponding category",
  "years_of_experience": Number of years of experience,
  "skills": ["Skill 1", "Skill 2", "Skill 3"],
  "email": "Candidate's email address",
  "phone": "Candidate's phone number",
  "linkedin": "Candidate's LinkedIn profile"
}
\end{lstlisting}
\end{tcolorbox}

    \begin{figure}[H]
        \centering
        \includegraphics[scale=0.7]{workflows/CV UseCase - v3.PNG}
        \caption{Zap de Zapier pour la classification automatique des tâches}
    \end{figure}
    
\section{Annexe 2 : Veille concurrentielle des offres télécoms}

\begin{tcolorbox}[colback=gray!5,colframe=gray!60!black,title={Prompt: Telecom Offer Comparison}, breakable]
You will compare two sets of commercial offers extracted from the websites of two telecom operators: \textbf{Ooredoo} and \textbf{Orange}.  
The data is already structured (or semi-structured) and contains, for each offer:

\begin{itemize}
    \item The name or title of the offer
    \item The price (promotional and/or standard, duration, conditions)
    \item Included services (data, calls, SMS, etc.)
    \item Options or bonuses
    \item Specific conditions (commitment, duration, fees, etc.)
\end{itemize}

Here are the fixed internet data from Ooredoo: \texttt{\{\{11.content[].text.value\}\}} \\
Here are the mobile internet data from Ooredoo: \texttt{\{\{31.content[].text.value\}\}} \\[0.5em]
Here are the fixed internet data from Orange: \texttt{\{\{19.content[].text.value\}\}} \\
And here are the mobile internet data from Orange: \texttt{\{\{32.content[].text.value\}\}}

\textbf{Your objective:} produce a clear and useful comparative analysis for competitive intelligence, including:

\begin{itemize}
    \item A comparison by segments (e.g., prepaid mobile, postpaid mobile, internet box, etc.)
    \item A highlight of the strengths and weaknesses of each operator
    \item A synthesis of price gaps and included advantages (data volume, unlimited calls, bonus services, etc.)
    \item An identification of similar or directly competing offers
    \item A suggestion of potential competitive advantages to monitor
\end{itemize}

\textbf{Expected output:}
\begin{itemize}
    \item A comparative text by offer type
    \item An analytical summary (5–10 key points) highlighting the most strategic differences
    \item If possible, an overall score or rating per segment for each operator  
          (e.g., price competitiveness, offer richness, value for money, etc.)
\end{itemize}
\end{tcolorbox}

    
\section{Annexe 3 : Priorisation et assignation automatique des tâches}


\begin{tcolorbox}[colback=gray!5,colframe=gray!60!black,title={Prompt Description}, breakable]
You are a project-management assistant for an e-commerce team.

Given a task description, do the following:
\begin{enumerate}
    \item Categorize the task as one of: \texttt{support}, \texttt{quality\_issue}, or \texttt{new\_feature}.
    \item Assess the priority: \texttt{low}, \texttt{medium}, \texttt{high}, or \texttt{critical}.
    \item Choose the most appropriate team member based on the task type and priority.
\end{enumerate}

Return a JSON object with exactly these keys:
\begin{itemize}
    \item \texttt{"type"}: the task category
    \item \texttt{"priority"}: the priority level
    \item \texttt{"assigned\_to"}: the \textbf{role tag only} — must be one of  
    \texttt{frontend\_dev}, \texttt{backend\_dev}, \texttt{qa\_engineer}, \texttt{support\_senior}, \texttt{support\_junior}
    \item \texttt{"name"}: the full name of the assigned person
    \item \texttt{"email"}: their email address
    \item \texttt{"trello\_member\_id"}: their Trello member ID
\end{itemize}

\textbf{Team directory (use only these values):}

\begin{itemize}
    \item \textbf{Mohamed} — \texttt{frontend\_dev} \\
    Email: \texttt{ihebbcl@gmail.com} \\
    Trello ID: \texttt{683d710046487a2d17c11e35}

    \item \textbf{Cheikh} — \texttt{backend\_dev} \\
    Email: \texttt{ihebih99@gmail.com} \\
    Trello ID: \texttt{683d70ff37a63bac3d5ac7e0}

    \item \textbf{Sonia} — \texttt{qa\_engineer} \\
    Email: \texttt{soniaghar@gmail.com} \\
    Trello ID: \texttt{683d710275957a589dcc68e7}

    \item \textbf{Amal} — \texttt{support\_senior} \\
    Email: \texttt{sonia.gharsalli@tek-up.tn} \\
    Trello ID: \texttt{61570095e0cdea36d1f9b425}

    \item \textbf{Maram} — \texttt{support\_junior} \\
    Email: \texttt{maramtrabelsi1212@gmail.com} \\
    Trello ID: \texttt{67f3b82b43d8d62b57567e2a}
\end{itemize}

Respond \textbf{only} with the JSON, no extra text.

\textbf{Example format:}
\begin{lstlisting}[language=json,frame=single,basicstyle=\ttfamily\footnotesize]
{
  "type": "support",
  "priority": "high",
  "assigned_to": "support_senior",
  "name": "Amal",
  "email": "sonia.gharsalli@tek-up.tn",
  "trello_member_id": "61570095e0cdea36d1f9b425"
}
\end{lstlisting}

\textbf{Here is the task to analyze:}

Subject: \{\{ \$json.subject \}\} \\
Content: \{\{ \$json.text \}\}
\end{tcolorbox}

    \begin{figure}[H]
        \centering
        \includegraphics[scale=0.7]{workflows/Automated Task Priotization.png}
        \caption{Workflow n8n pour la Priorisation et assignation automatique des tâches}
    \end{figure}
    
\section{Annexe 4 : Transcription intelligente des réunions et extraction automatique des actions}

\begin{tcolorbox}[colback=gray!5,colframe=gray!60!black,title=Prompt – Meeting Summary Extraction, breakable]
\small

You are a smart productivity assistant. Based on the following meeting transcript, you must extract:

\begin{enumerate}
  \item A concise meeting summary (maximum 5 lines), including for each team member their assigned tasks and due dates if explicitly mentioned.
  \item A list of decisions made during the meeting.
  \item A flat list of action items in structured JSON format with the following fields:
  \begin{itemize}
    \item \textbf{subject}: a title summarizing the meeting purpose or topic
    \item \textbf{summary}: meeting summary as described above
    \item \textbf{decision\_X}: each decision as a separate key (e.g., decision\_1, decision\_2, etc.)
    \item \textbf{title\_X}: short title of the task
    \item \textbf{description\_X}: what needs to be done
    \item \textbf{assignee\_full\_name\_X}: full name of the responsible person
    \item \textbf{assignee\_email\_X}: their email address
    \item \textbf{assignee\_trello\_id\_X}: their Trello member ID
    \item \textbf{due\_date\_X}: deadline (ISO format \texttt{YYYY-MM-DD}), or \texttt{null} if vague
    \item \textbf{priority\_X}: high, medium, or low
  \end{itemize}
\end{enumerate}

Use this team reference to match names, emails, and Trello IDs:

\begin{itemize}
  \item \textbf{Mohamed} (frontend\_dev)
    \begin{itemize}
      \item Email: \texttt{ihebbcl@gmail.com}
      \item Trello ID: \texttt{683d710046487a2d17c11e35}
    \end{itemize}
  \item \textbf{Josef} (backend\_dev)
    \begin{itemize}
      \item Email: \texttt{ihebih99@gmail.com}
      \item Trello ID: \texttt{683d70ff37a63bac3d5ac7e0}
    \end{itemize}
  \item \textbf{Sonia} (qa\_engineer)
    \begin{itemize}
      \item Email: \texttt{soniaghar@gmail.com}
      \item Trello ID: \texttt{683d710275957a589dcc68e7}
    \end{itemize}
  \item \textbf{Amal} (support\_senior)
    \begin{itemize}
      \item Email: \texttt{sonia.gharsalli@tek-up.tn}
      \item Trello ID: \texttt{61570095e0cdea36d1f9b425}
    \end{itemize}
  \item \textbf{Maram} (support\_junior)
    \begin{itemize}
      \item Email: \texttt{maramtrabelsi1212@gmail.com}
      \item Trello ID: \texttt{67f3b82b43d8d62b57567e2a}
    \end{itemize}
\end{itemize}

Here is the transcript:
\begin{quote}
\texttt{{\{\{ \$json.text \}\}}}
\end{quote}

Return only a valid flat JSON object using the following structure:

\begin{verbatim}
{
  "subject": "Meeting Recap - [custom subject or date]",
  "summary": "[Insert summary here, including team tasks and due dates if any]",
  "decision_1": "First decision",
  "decision_2": "Second decision",
  "title_1": "Write the Q2 report",
  "description_1": "Gather sales data and draft the Q2 report for review.",
  "assignee_full_name_1": "Mohamed",
  "assignee_email_1": "ihebbcl@gmail.com",
  "assignee_trello_id_1": "683d710046487a2d17c11e35",
  "due_date_1": "2025-06-15",
  "priority_1": "high",
  "title_2": "...",
  "description_2": "...",
  "assignee_full_name_2": "...",
  ...
}
\end{verbatim}

\end{tcolorbox}

    \begin{figure}[H]
        \centering
        \includegraphics[scale=0.7]{workflows/Smart Meeting Transcription.png}
        \caption{Workflow n8n pour la Transcription intelligente des réunions et extraction automatique des actions}
    \end{figure}
    
\section*{Annexe 5 : Vérification automatique de la conformité des projets}

\begin{tcolorbox}[colback=gray!5,colframe=gray!60!black,title={Prompt used: Compliance Analysis of Software Development Project}, breakable]
You are a \textbf{smart compliance analysis agent} reviewing a software development project.

You will be given:
\begin{enumerate}
    \item A list of all features to be implemented, from the main program.
    \item Detailed team reports from the following five teams: Frontend, Backend, DevOps, AI/ML, and QA.
\end{enumerate}

Your task is to check whether \textbf{each feature} from the main program is implemented, and if so, by which team, and whether that team is the correct one.

\medskip
\textbf{Here is the main program feature list:}
\begin{verbatim}
{{ $('Code6').item.json.formatted_main_program_features }}
\end{verbatim}

\textbf{Here are the detailed team reports:}
\begin{verbatim}
{{ $('Code6').item.json.spec_detaillee }}
\end{verbatim}

You must return a \textbf{strictly valid JSON object} with the following structure:

\begin{lstlisting}[language=json,frame=single,basicstyle=\ttfamily\footnotesize]
{
  "compliance_report": {
    "features_status": [
      {
        "feature": "string",
        "assigned_team": "string",
        "status": "valid"
      }
    ],
    "missing_features": ["string"],
    "duplicated_features": [
      {
        "feature": "string",
        "actual_teams": "string"
      }
    ],
    "team_mismatches": [
      {
        "feature": "string",
        "actual_team": "string",
        "expected_team": "string"
      }
    ]
  }
}
\end{lstlisting}

\textbf{Rules:}

For each feature in the main program:

\begin{itemize}
    \item Set status to:
    \begin{itemize}
        \item \texttt{"valid"} if the feature is implemented by the correct team.
        \item \texttt{"missing"} if not implemented by any team.
        \item \texttt{"duplicated"} if implemented by more than one team.
        \item \texttt{"team\_mismatch"} if implemented by the wrong team.
    \end{itemize}
\end{itemize}

The expected team for each feature should be inferred from the standard responsibilities of each team:

\begin{itemize}
    \item \textbf{Frontend:} UI components, user interactions, authentication flows, dashboards, visualizations, mobile responsiveness, in-app messaging, localization, image/video handling.
    \item \textbf{Backend:} APIs, database, encryption, access control, microservices, logs, error tracking, backend logic, secure communication, backups, audit trails.
    \item \textbf{DevOps:} CI/CD, deployments, load balancing, automated testing suites, monitoring.
    \item \textbf{AI/ML:} AI search, recommendations, predictive analytics.
    \item \textbf{QA:} feature validation, testing plans, review/feedback systems, report generation.
\end{itemize}

Under \texttt{duplicated\_features}, list the duplicated feature and the teams that both implemented it.  
Under \texttt{team\_mismatches}, report for any team mismatch: which team implemented it and which team was expected.

\medskip
Your response must be a \textbf{strictly valid JSON} with no comments and no trailing commas.
\end{tcolorbox}

    \begin{figure}[H]
        \centering
        \includegraphics[scale=0.7]{workflows/Conformité.png}
        \caption{Workflow n8n pour la Vérification automatique de la conformité des projets}
    \end{figure}