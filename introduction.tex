\chapter*{Introduction générale}
\addcontentsline{toc}{chapter}{Introduction générale} % to include the introduction to the table of content
\markboth{Introduction générale}{} %To redefine the section page head

Dans un contexte où les entreprises cherchent à accroître leur efficacité opérationnelle et à réduire les tâches répétitives, l’automatisation des processus métiers s’impose comme une nécessité stratégique. L’essor des technologies d’intelligence artificielle (IA) et des plateformes d’orchestration de flux comme n8n offre aujourd’hui de nouvelles opportunités pour optimiser la gestion quotidienne des projets et améliorer la productivité des équipes.

Les organisations modernes génèrent un volume considérable d’informations issues de différents canaux : courriels, réunions, outils de support, plateformes de gestion de projets, et bases documentaires. Le traitement manuel de ces flux d’informations entraîne souvent des retards, des erreurs et un manque de cohérence dans la priorisation des tâches. Face à ces défis, l’automatisation intelligente — combinant IA générative, intégration d’outils collaboratifs et logique de workflow — devient un levier majeur pour renforcer l’agilité des entreprises.

C’est dans ce cadre que s’inscrit le projet intitulé « Développement d’agents d’IA pour l’automatisation de cas d’utilisation concrets en entreprise ». Ce projet a pour objectif de concevoir et de développer des systèmes d’automatisation capables d’interpréter et de traiter les données issues de différentes sources (transcriptions de réunions, tickets de support, documents techniques, etc.), afin de générer automatiquement des tâches, les classer, les assigner aux bons responsables et les synchroniser avec des outils tels que Trello, Gmail ou Google Drive.

Le projet repose sur une architecture modulaire intégrant des modèles de langage avancés (LLMs), un moteur d’orchestration n8n, et des connecteurs API permettant la communication entre les différents services. Il vise à démontrer comment l’automatisation augmentée par l’IA peut transformer la gestion des flux d’information en un processus fluide, cohérent et intelligent.

La start-up Deep-Shift AI Academy, spécialisée dans les solutions IA et d’automatisation, a soutenu la mise en œuvre de ce projet, qui s’inscrit dans une démarche d’amélioration continue et d’innovation au service de la productivité des équipes.

Ce rapport est structuré en plusieurs chapitres, organisés de manière à retracer l’évolution du projet depuis sa conception jusqu’à sa mise en œuvre finale :

\begin{itemize}
    \item Le premier chapitre présente le cadre général du projet, incluant la présentation de l’organisme d’accueil, la problématique, les objectifs, l’état de l’art et la méthodologie de travail adoptée, notamment l’approche Scrum.
    
    \item Le deuxième chapitre est consacré à l’analyse et à la spécification des besoins. Il décrit les exigences fonctionnelles et non fonctionnelles du système, les cas d’utilisation, l’architecture générale de la plateforme n8n, ainsi que l’environnement technique de développement.
    
    \item Le troisième chapitre regroupe les premiers cas d’utilisation développés dans les domaines des ressources humaines et du marketing. Il présente les workflows conçus pour la classification automatique des CV et la veille concurrentielle des offres télécoms.
    
    \item Le quatrième chapitre est dédié aux cas d’utilisation orientés vers la gestion de projet. Il décrit les workflows permettant la priorisation et l’assignation automatique des tâches, l’extraction d’actions à partir de transcriptions de réunions, ainsi que la vérification automatique de la conformité des projets.
    
    \item Le cinquième chapitre présente le dernier cas d’utilisation, portant sur le développement d’un chatbot Rasa de dépannage pour un fournisseur d’accès Internet, intégrant des capacités d’analyse et de diagnostic automatisé.
    
    \item Enfin, la conclusion générale récapitule les résultats obtenus, les principales contributions du projet et ouvre sur des perspectives d’amélioration, notamment l’intégration de la voix et des agents conversationnels intelligents dans les workflows d’automatisation.
\end{itemize}

Ainsi, ce projet illustre comment l’automatisation intelligente, fondée sur l’intelligence artificielle et l’orchestration de workflows, peut redéfinir la gestion opérationnelle moderne, en plaçant la donnée et la collaboration au cœur de l’efficacité organisationnelle.
